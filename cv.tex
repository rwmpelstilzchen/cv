\documentclass[11pt,a4paper]{moderncv}        % possible options include font size ('10pt', '11pt' and '12pt'), paper size ('a4paper', 'letterpaper', 'a5paper', 'legalpaper', 'executivepaper' and 'landscape') and font family ('sans' and 'roman')

% \tolerance=99999
\usepackage{microtype}

\usepackage[unicode]{hyperref}

% moderncv themes
\moderncvstyle{banking}                             % style options are 'casual' (default), 'classic', 'oldstyle' and 'banking'
\renewcommand*{\makeheaddetailssymbol}{\qquad}

\moderncvcolor{blue}                               % color options 'blue' (default), 'orange', 'green', 'red', 'purple', 'grey' and 'black'
% \definecolor{turonBlue}{HTML}{004C7F}
% \definecolor{darkscarlet}{rgb}{0.34, 0.01, 0.1}
% \definecolor{Burgundy}{RGB}{144, 0, 32}
% \definecolor{color0}{rgb}{0,0,0}% black
\definecolor{color1}{HTML}{004C7F}% light blue
% \definecolor{color2}{rgb}{1.0, 0.01, 0.1}% dark grey
\renewcommand*{\namefont}{\Huge\mdseries\upshape}
%\renewcommand{\familydefault}{\sfdefault}         % to set the default font; use '\sfdefault' for the default sans serif font, '\rmdefault' for the default roman one, or any tex font name
%\nopagenumbers{}                                  % uncomment to suppress automatic page numbering for CVs longer than one page
\patchcmd{\makehead}
  {\setlength{\makeheaddetailswidth}{0.8\textwidth}}
  {\setlength{\makeheaddetailswidth}{0.7\textwidth}}
  {}
  {}
\makeatletter
\patchcmd{\makehead}{%search
  \flushmakeheaddetails\@firstmakeheaddetailselementtrue\\\null}{%replace
  \flushmakeheaddetails\@firstmakeheaddetailselementtrue\par\vspace{-0.9\baselineskip}\null}{%success
  }{%failure
  }
\makeatother

%\usepackage[scale=0.75]{geometry}
\usepackage{geometry}
%\setlength{\hintscolumnwidth}{3cm}                % if you want to change the width of the column with the dates
%\setlength{\makecvtitlenamewidth}{10cm}           % for the 'classic' style, if you want to force the width allocated to your name and avoid line breaks. be careful though, the length is normally calculated to avoid any overlap with your personal info; use this at your own typographical risks...

% Wait for Biblatex 2; http://tex.stackexchange.com/questions/55256
%\DeclareSortingScheme{noneyear}{
% \sort{\citeorder}
% \sort{\field{year}}
%}

\usepackage{marginnote,needspace}

\makeatletter
\renewcommand\sectionfont{\color{color1}\large}
\renewcommand*{\sectionstyle}[1]{{%
	\needspace{1\baselineskip}%
	% \reversemarginpar\marginnote{\fontspec{Symbola}⋄}\normalmarginpar%
	\hspace{-0.5em}%
	{\sectionfont\MakeLowercase{\textsc{#1}}}}}
\renewcommand*{\section}[1]{%
  \par\addvspace{3.5ex}%
  \phantomsection{}% reset the anchor for hyperrefs
  \addcontentsline{toc}{section}{#1}%
  \sectionstyle{#1}%
  \par\nobreak\addvspace{1.0ex}\@afterheading}

\renewcommand\subsectionfont{\color{color1}}
\renewcommand*{\subsectionstyle}[1]{{%
	\parbox[t]{\textwidth}{\subsectionfont\textit{#1}}}}
\renewcommand*{\subsection}[1]{%
  \addvspace{1.5ex}%
  \phantomsection{}% reset the anchor for hyperrefs
  \addcontentsline{toc}{subsection}{#1}%
  \subsectionstyle{#1}%
  \par\nobreak\addvspace{2.0ex}\@afterheading}
\makeatother


\usepackage[backend=biber, style=authoryear, eventdate=comp, sorting=none]{biblatex}
\bibliography{publications}
\renewcommand{\mkbibnamefamily}[1]{\textsc{#1}}
\renewcommand{\subtitlepunct}{:\ }


\usepackage[silent]{fontspec}
\usepackage[silent,nil,bidi=basic]{babel}

\babelprovide[import=en-GB,main]{english}
\babelprovide[import=cy]{welsh}
\babelprovide[import=he]{hebrew}

% \WarningFilter{fontspec}{is using the default features for language}

\babelfont[hebrew]{rm}[Script=Hebrew, ItalicFont={Days-Nights}, ItalicFeatures={Scale=0.95}]{RacRaze}

\usepackage{fontspec}
%\usepackage{xltxtra}
% \usepackage{bidi}
% \babelfont{rm}[
		% ItalicFont = {Gentium Plus Italic},
		% BoldFont       = {Gentium Basic Bold},
		% BoldFeatures   = {%
			% SmallCapsFont=JuniusX,
			% SmallCapsFeatures={Letters=SmallCaps},
			% SmallCapsFeatures={FakeBold=2},
		% },
		% BoldItalicFont = {Gentium Basic Bold Italic},
		% BoldItalicFeatures   = {%
			% SmallCapsFont=JuniusX,
			% SmallCapsFeatures={Letters=SmallCaps},
			% SmallCapsFeatures={FakeBold=2},
		% },
	% ]{Gentium Plus}
\setmainfont{Gentium Plus}
%\setmainfont{Gentium Book Basic}
\setmonofont[Scale=0.9]{PragmataPro}

% \newcommand{\Hebrew}[1]{\bgroup\fontspec[Script=Hebrew, ItalicFont={Days-Nights}, ItalicFeatures={Scale=0.95}]{Rutz_OE} \RL{#1}\egroup}
% \newcommand{\BiblicalHebrew}[1]{\bgroup\fontspec[Script=Hebrew]{SBL Hebrew} \RL{#1}\egroup}
\newcommand{\BiblicalHebrew}[1]{\bgroup\fontspec[Script=Hebrew, Renderer=Harfbuzz]{SBL Hebrew}\textdir TRT #1\egroup}
\newcommand{\Hebrew}[1]{\BiblicalHebrew{#1}}
\newcommand{\texfont}{\fontspec{Palatino Linotype}}

\usepackage[super]{nth}

\newcommand{\foreign}[1]{\emph{#1}}

\name{Júda}{Ronén\vspace{0.5ex}}
%\title{}                               % optional, remove / comment the line if not wanted
\address{44/C/4 Mekor Ĥajim St.}{Jerusalem, 9346552}{Israeli Post}% optional, remove / comment the line if not wanted; the "postcode city" and "country" arguments can be omitted or provided empty
%\phone[mobile]{+972-54-6509361}                   % optional, remove / comment the line if not wanted; the optional "type" of the phone can be "mobile" (default), "fixed" or "fax"
\phone[mobile]{+972-52-5587769}                   % optional, remove / comment the line if not wanted; the optional "type" of the phone can be "mobile" (default), "fixed" or "fax"
\phone[fixed]{+972-2-6419913}
\email{foo@digitalwords.net}
\homepage{me.digitalwords.net}
%\social[github]{rwmpelstilzchen}
\social[stackexchange]{3978580}
%\extrainfo{additional information}                 % optional, remove / comment the line if not wanted
%\photo[64pt][0.4pt]{qr.png}                       % optional, remove / comment the line if not wanted; '64pt' is the height the picture must be resized to, 0.4pt is the thickness of the frame around it (put it to 0pt for no frame) and 'picture' is the name of the picture file
%\quote{Some quote}                                 % optional, remove / comment the line if not wanted


\newcommand{\cvdate}{2015·7·25}
\fancyfoot[l]{\color{color2}[\cvdate]}

\begin{document}

\makecvtitle

\section{Education}
\cventry{2008–2015}{M.A.\ in Linguistics}{Hebrew University}{Jerusalem}{}{}  % arguments 3 to 6 can be left empty
\cventry{2004–2007}{B.A.\ in Linguistics and B.Sc.\ in Computer Science}{Hebrew University}{Jerusalem}{}{}
\cventry{2009}{Ysgol Haf Dysgu Cymraeg (Welsh language summer school)}{Coleg Meirion-Dwyfor}{Pwllheli}{}{}

\section{Master thesis}
\cvitem{Title}{\emph{%
	%{se ðe ís soð wysdom~· {\fontspec{Junicode}⁊} sawla líf}
	A Textual-Structural Study of the demonstrative pronouns in Old English:
	classification, characterization and description
	of the inventory shown in Ælfric’s \emph{Lives of Saints}.}}
\cvitem{Advisor}{Nimrod \textsc{Barri}.}
\cvitem{Abstract}{
Description of the syntactical structures in Old English, in which the demonstrative pronouns (\textit{se} and \textit{ðis}) occur independently of a following nominal phrase: structural classification, characterization and analysis of their internal structure and their function in the text by a comprehensive examination of the inventory shown in the corpus, Ælfric’s \emph{Lives of Saints}.
}

\section{Academic interests}
\cvlistdoubleitem{Text linguistics}{Linguistic analysis of translation}
\cvlistdoubleitem{Linguistic analysis of narrative}{Welsh, Old~English and Japanese}
\cvlistdoubleitem{General linguistics}{Classification systems}



\section{Conference presentations}

\cventry{2015}
	{llygaid [sydd] ganddynt, ac ni welant:}
	{XV International Congress of Celtic Studies}
	{Glasgow}
	{}
	{\emph{Mediating Senses through Translation Choices}}

\cventry{2012}
	{\fontspec{Gentium Book Basic}na{\fontspec{Gentium}ʿ}ăśɛ wə-nišmå{\fontspec{Gentium}ʿ}:}
	{Studientag in the Honour of Ariel Shisha-Halevy}
	{Jesrusalem}
	{}
	{\emph{a linguistic study of a translator’s choices — the verb \emph{šm{\fontspec{Gentium}ʿ}} in the Welsh translation of the Pentateuch}. Presented in Hebrew (\Hebrew{\BiblicalHebrew{נַעֲשֶׂ֥ה וְנִשְׁמָֽע}: עיון בלשני בבחירותיו של מתרגם — הפועל „\BiblicalHebrew{שָׁמַע}” בתרגום התנ״ך הוולשי}) at a \emph{Studientag} in the honour of Ariel Shisha-Halevy, on the occasion of his retirement; the Hebrew University of Jerusalem, the Department of Linguistics. (Invited).}

\cventry{2011}
	{Ni a wnawn, ac a wrandawn:}
	{XIV International Congress of Celtic Studies}
	{Maigh Nuad}
	{}
	{\emph{William Morgan's Choices in His 1588 Welsh Translation of the Pentateuch — the Case of the Hebrew Verb \emph{šm{\fontspec{Gentium}ʿ}}} (“hear”).}

\cventry{2009}
	{Between Narrative and Dialogue:}
	{Cynhadledd Ryngwladol Edward Lhuyd}
	{Aberystwyth}
	{}
	{\emph{Syntactical Features of Signalling Speech within Narrative in Modern Welsh.}}



\section{Research projects}

\cventry{2008–2010}{Šenoute's Rhetorical Syntax}{Hebrew University of Jerusalem / Israel Science Foundation}{Jerusalem}{}{Assistant to Ariel \textsc{Shisha-Halevy}.}



\section{Awards}

\cventry{2008}
	{Fraenkel award for outstanding papers in linguistics}
	{Hebrew University of Jerusalem}
	{Jerusalem}
	{}
	{Paper title: \emph{Between Narrative and Dialogue~— syntactical features of speech within narrative in Modern Welsh} (written in Hebrew); advisor: Ariel Shisha-Halevy.}



\section{Publications}

\nocite{*}
\printbibliography[heading=none]



\section{Employment}

\cventry{2007–}
	{Webmaster}
	{Ofek School for Gifted Children}
	{Jerusalem}
	{}
	{Developing, designing and maintaining a website and school web applications (course selection and feedback).}

\cventry{2007–2009}
	{Typesetter}
	{Journal d'Analyse Mathématique and Israel Journal of Mathematics}
	{Jerusalem}
	{}
	{Typesetting and correcting proofread articles in mathematics}



\section{Language skills}

\cvitem{Fluent}
	{Hebrew (native), English}
\cvitem{Reading knowledge}
	{Modern and Middle Welsh, Old~English, Esperanto, Japanese, Coptic (Sahidic and Bohairic dialects), German}
\cvitem{Basic knowledge}
	{Pre-Coptic Egyptian (Late, Middle and Old), Old~Icelandic, Irish, Greek, Latin, Gothic, Mongolian (Khalkha dialect)}



\section{Computer skills}

\cvdoubleitem
	{Typesetting}{{\texfont\LaTeX}}
	{Text editing}{Vim}
\cvdoubleitem
	{Programming languages}{Python, C/C++, PHP, Java}
	{Operating systems}{(Arch, Gentoo) Linux, FreeBSD, Mac~OS~X}
\cvdoubleitem
	{Markup and style languages}{HTML, CSS}
	{Database systems}{MySQL, Oracle DBMS}
\cvdoubleitem
	{Web publishing}{Pelican, Jekyll, MediaWiki, Wordpress}
	{}{}


\section{Music skills}

\cvitem{Plucked string instruments}{Mandolin, guitar/lute, ukulele}
\cvitem{Fipple flutes}{Irish whistle, ocarina, recorder}
\cvitem{Keyboard}{Piano, Melodica}

\end{document}
