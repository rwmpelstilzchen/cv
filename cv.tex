\documentclass[11pt,a4paper]{moderncv}        % possible options include font size ('10pt', '11pt' and '12pt'), paper size ('a4paper', 'letterpaper', 'a5paper', 'legalpaper', 'executivepaper' and 'landscape') and font family ('sans' and 'roman')

% \tolerance=99999
\usepackage{microtype}

\usepackage[unicode]{hyperref}

% moderncv themes
\moderncvstyle{banking}                             % style options are 'casual' (default), 'classic', 'oldstyle' and 'banking'
\renewcommand*{\makeheaddetailssymbol}{\qquad}

\moderncvcolor{blue}                               % color options 'blue' (default), 'orange', 'green', 'red', 'purple', 'grey' and 'black'
% \definecolor{turonBlue}{HTML}{004C7F}
% \definecolor{darkscarlet}{rgb}{0.34, 0.01, 0.1}
% \definecolor{Burgundy}{RGB}{144, 0, 32}
% \definecolor{color0}{rgb}{0,0,0}% black
\definecolor{color1}{HTML}{004C7F}% light blue
% \definecolor{color2}{rgb}{1.0, 0.01, 0.1}% dark grey
\renewcommand*{\namefont}{\Huge\mdseries\upshape}
%\renewcommand{\familydefault}{\sfdefault}         % to set the default font; use '\sfdefault' for the default sans serif font, '\rmdefault' for the default roman one, or any tex font name
%\nopagenumbers{}                                  % uncomment to suppress automatic page numbering for CVs longer than one page
\patchcmd{\makehead}
  {\setlength{\makeheaddetailswidth}{0.8\textwidth}}
  {\setlength{\makeheaddetailswidth}{0.7\textwidth}}
  {}
  {}
\makeatletter
\patchcmd{\makehead}{%search
  \flushmakeheaddetails\@firstmakeheaddetailselementtrue\\\null}{%replace
  \flushmakeheaddetails\@firstmakeheaddetailselementtrue\par\vspace{-0.9\baselineskip}\null}{%success
  }{%failure
  }
\makeatother

%\usepackage[scale=0.75]{geometry}
\usepackage{geometry}
%\setlength{\hintscolumnwidth}{3cm}                % if you want to change the width of the column with the dates
%\setlength{\makecvtitlenamewidth}{10cm}           % for the 'classic' style, if you want to force the width allocated to your name and avoid line breaks. be careful though, the length is normally calculated to avoid any overlap with your personal info; use this at your own typographical risks...

% Wait for Biblatex 2; http://tex.stackexchange.com/questions/55256
%\DeclareSortingScheme{noneyear}{
% \sort{\citeorder}
% \sort{\field{year}}
%}

\usepackage{marginnote,needspace}

\makeatletter
\renewcommand\sectionfont{\color{color1}\large}
\renewcommand*{\sectionstyle}[1]{{%
	\needspace{1\baselineskip}%
	% \reversemarginpar\marginnote{\fontspec{Symbola}⋄}\normalmarginpar%
	\hspace{-0.5em}%
	{\sectionfont\MakeLowercase{\textsc{#1}}}}}
\renewcommand*{\section}[1]{%
  \par\addvspace{3.5ex}%
  \phantomsection{}% reset the anchor for hyperrefs
  \addcontentsline{toc}{section}{#1}%
  \sectionstyle{#1}%
  \par\nobreak\addvspace{1.0ex}\@afterheading}

\renewcommand\subsectionfont{\color{color1}}
\renewcommand*{\subsectionstyle}[1]{{%
	\parbox[t]{\textwidth}{\subsectionfont\textit{#1}}}}
\renewcommand*{\subsection}[1]{%
  \addvspace{1.5ex}%
  \phantomsection{}% reset the anchor for hyperrefs
  \addcontentsline{toc}{subsection}{#1}%
  \subsectionstyle{#1}%
  \par\nobreak\addvspace{2.0ex}\@afterheading}
\makeatother


\usepackage[backend=biber, style=authoryear, eventdate=comp, sorting=none]{biblatex}
\bibliography{publications}
\renewcommand{\mkbibnamefamily}[1]{\textsc{#1}}
\renewcommand{\subtitlepunct}{:\ }


\usepackage[silent]{fontspec}
\usepackage[silent,nil,bidi=basic]{babel}

\babelprovide[import=en-GB,main]{english}
\babelprovide[import=cy]{welsh}
\babelprovide[import=he]{hebrew}

% \WarningFilter{fontspec}{is using the default features for language}

\babelfont[hebrew]{rm}[Script=Hebrew, ItalicFont={Days-Nights}, ItalicFeatures={Scale=0.95}]{RacRaze}

\usepackage{fontspec}
%\usepackage{xltxtra}
% \usepackage{bidi}
% \babelfont{rm}[
		% ItalicFont = {Gentium Plus Italic},
		% BoldFont       = {Gentium Basic Bold},
		% BoldFeatures   = {%
			% SmallCapsFont=JuniusX,
			% SmallCapsFeatures={Letters=SmallCaps},
			% SmallCapsFeatures={FakeBold=2},
		% },
		% BoldItalicFont = {Gentium Basic Bold Italic},
		% BoldItalicFeatures   = {%
			% SmallCapsFont=JuniusX,
			% SmallCapsFeatures={Letters=SmallCaps},
			% SmallCapsFeatures={FakeBold=2},
		% },
	% ]{Gentium Plus}
\setmainfont{Gentium Plus}
%\setmainfont{Gentium Book Basic}
\setmonofont[Scale=0.9]{PragmataPro}

% \newcommand{\Hebrew}[1]{\bgroup\fontspec[Script=Hebrew, ItalicFont={Days-Nights}, ItalicFeatures={Scale=0.95}]{Rutz_OE} \RL{#1}\egroup}
% \newcommand{\BiblicalHebrew}[1]{\bgroup\fontspec[Script=Hebrew]{SBL Hebrew} \RL{#1}\egroup}
\newcommand{\BiblicalHebrew}[1]{\bgroup\fontspec[Script=Hebrew, Renderer=Harfbuzz]{SBL Hebrew}\textdir TRT #1\egroup}
\newcommand{\Hebrew}[1]{\BiblicalHebrew{#1}}
\newcommand{\texfont}{\fontspec{Palatino Linotype}}

\usepackage[super]{nth}

\newcommand{\foreign}[1]{\emph{#1}}


\begin{document}
%\setRL\frenchspacing\HebrewFont\setfootnoteRL
\pagestyle{empty}

\begin{center}
	%\fontspec{Gentium}
	%\fontspec{Palatino}
	\fontspec{Palatino Linotype}
	\addfontfeatures{Color=203040FF}
	\huge \textsc{Curriculum Vitæ}
\end{center}

\bigskip
\bigskip

\reversemarginpar
%\marginnote{~~~\includegraphics[scale=0.5]{QR.png}}
\marginnote{\vskip-1em~~~\includegraphics[scale=0.5]{QR.png}}
%\marginnote{\vskip-0.5em~~~\includegraphics[scale=0.5]{QRcontact.png}}
\begin{tabular}{ll}
	\textbf{\ruby{{\fontspec{Fontin}\addfontfeatures{Color=203040FF}Júda Ronén}}{\fontsize{6}{6}\fontspec{Gentium}[ˈjudă ʁoˈnen]}}\\
	Telephone:&\texttt{+972-2-6419913}\\
	Email:&\url{foo@digitalwords.net}\\
	Website:&\url{me.digitalwords.net}\\
	Public key:&\url{digitalwords.net/public.txt}\\
	~
\end{tabular}
\hfill
\begin{tabular}{l}
	4/6 \ruby{Ha Náxal}{(\Hebrew{הנחל})} St.\\
	Ha Giv{\fontspec{Junicode}ʾ}á ha Carfatít\\
	Jerusalem\\
	97882 (Israeli post)
\end{tabular}

\bigskip

\begin{ressec}{Education}
\begin{resdescription}
	%\resdescitem{2012–present}{\emph{Ph.D.\ in Linguistics} at the Hebrew University of Jerusalem\newline
%		Dissertation title: ‘’ (provisional); advisor: Ariel Shisha-Halevy}
	\resdescitem{2007–present}{\emph{M.A.\ in Linguistics} at the Hebrew University of Jerusalem\newline
		Thesis title: ‘\textit{\fontspec{Junicode}se ðe ís soð ƿysdom · ⁊ saƿla líf}~— A Textual-Structural Study of \textit{\fontspec{Junicode}se,\,seo,\,þæt} in Old English: classification, characterization and description of the inventory shown in Ælfric's \textit{Lives of Saints} (provisional);
		advisors: Nimrod Barri and Ariel Shisha-Halevy}
%	Thesis title: ‘The Narrative Structure of Ælfric's Vitæ Sanctorum’ (provisional); advisors: Nimrod Barri and Ariel Shisha-Halevy}
	\resdescitem{2004–2007}{\emph{B.A.\ in Linguistics} and \emph{B.Sc.\ in Computer Science} at the Hebrew University of Jerusalem}
\end{resdescription}
\end{ressec}

\begin{ressec}{Academic interests}
\begin{resitemize}
	\resitem Text linguistics\\
	\resitem Linguistic analysis of narrative\\
	\resitem General linguistics\\
	\resitem Linguistic analysis of translation\\
%	\resitem Interaction between textemes\\
	\resitem Welsh, Old~English, Coptic and Japanese syntax\\
%	\resitem Computational linguistics\\
	\resitem Classification systems and logographic writing systems
\end{resitemize}
\end{ressec}

~

\begin{ressec}{Publications}
	\begin{resdescription}
			\resdescitem{2012}{\emph{Collaborative Digital Edition of Ælfric's \textit{Lives of Saints}} (main contributor; in progress) {\url{[en.wikisource.org/wiki/Ælfric's_Lives_of_Saints]}}
		}
		\resdescitem{2010}{\emph{Babel: Journal of Translation from the World's Languages} (in Hebrew:
			\Hebrew{בָּבֶל: כְּתַב־עֵת לְתִרְגּוּם מִלְּשׁוֹנוֹת הָעוֹלָם})
			\mbox{\url{[bbl.digitalwords.net]}}
		}
%		\resdescitem{2010}{A Hebrew translation of the hitherto untranslated essays from \textit{Anarchism and Other Essays} (\Hebrew{אנרכיזם ומסות אחרות}) by Emma Goldman, including the biographic sketch. Ideologically self-published in the public domain (\url{bit.ly/***}).}
%		\resdescitem{2008}{Between Narrative and Dialogue: Syntactical Features of Speech within Narrative in Modern Welsh, \textit{Journal of Celtic Linguistics} 12}
	\end{resdescription}
\end{ressec}

\begin{ressec}{Research projects}
\begin{resdescription}
	\resdescitem{2008–10}{\emph{Šenoute's Rhetorical Syntax} (Hebrew University of Jerusalem / Israel Science Foundation), as an assistant to Ariel Shisha-Halevy}
\end{resdescription}
\end{ressec}


\begin{ressec}{Conference papers}
\begin{resdescription}
		\resdescitem{2012}{‘\textit{\fontspec{Palemonas MUFI}naʿăśɛ wə-nišmåʿ}: a Linguistic Study of a Translator's Choices — the Verb \textit{\fontspec{Palemonas MUFI}šmʿ} in the Welsh Translation of the Pentateuch’ (In Hebrew: \Hebrew{נַעֲשֶׂ֥ה וְנִשְׁמָֽע: עיון בלשני בבחירותיו של מתרגם — הפועל ’שָׁמַע‘ בתרגום התנ״ך הוולשי})\newline
			\emph{Studientag in the Honour of Ariel Shisha-Halevy, on the Occasion of His Retirement}, the Hebrew University of Jerusalem, the Department of Linguistics, Jerusalem, 29 May 2012 (invited)
		}
	\resdescitem{2011}{‘\textit{\fontspec{Junicode}Ni a wnawn, ac a wꝛandawn}: William Morgan's Choices in His 1588 Welsh Translation of the Pentateuch — the Case of the Hebrew Verb {\fontspec{Gentium}\textit{šmʿ}} (“hear”)’\newline
		\emph{XIV International Congress of Celtic Studies} (\emph{An 14ú Comhdháil Idirnáisiúnta sa Léann Ceilteach}) \mbox{\url{[celticstudiescongress.org]}}, Maigh Nuad, Aug 2011}
	\resdescitem{2009}{‘Between Narrative and Dialogue: Syntactical Features of Signalling Speech within Narrative in Modern Welsh’\newline
		\emph{Cynhadledd Ryngwladol Edward Lhuyd} \mbox{\url{[bit.ly/bEWKnV]}}, Aberystwyth, 1 Jul 2009}
\end{resdescription}
\end{ressec}


\begin{ressec}{Awards}
	\begin{resdescription}
		\resdescitem{2008}{\emph{Fraenkel award} for %excellent
		outstanding papers in linguistics (Hebrew University of Jerusalem)\newline
			Paper title: ‘Between Narrative and Dialogue: Syntactical Features of Speech within Narrative in Modern Welsh’ (in Hebrew); advisor: Ariel Shisha-Halevy}
	\end{resdescription}
\end{ressec}


\begin{ressec}{Employment}
\begin{resdescription}
	\resdescitem{2007–}{Developing, designing and maintaining a website and school web applications (course selection and feedback) for \emph{Ofek School for Gifted Children}, Jerusalem \mbox{\url{[ofek-gifted.org.il]}}}
	\resdescitem{2007–9}{Typesetting and correcting proofread articles in mathematics for the \emph{Journal d'Analyse Mathématique} \mbox{\url{[ma.huji.ac.il/jdm]}} and the \emph{Israel Journal of Mathematics} \mbox{\url{[ma.huji.ac.il/~ijmath]}}}
\end{resdescription}
\end{ressec}

\begin{ressec}{Languages}
\begin{resdescription}
	\resdescitem{Fluent:}{Hebrew (native), English}
	\resdescitem{Reading knowledge:}{Modern and Middle Welsh, Old~English, Esperanto, Japanese, Coptic (Sahidic and Bohairic dialects), German}
	\resdescitem{Basic knowledge:}{Pre-Coptic Egyptian (Late, Middle and Old), Old~Icelandic, Irish, Modern~Greek, Latin, Gothic, Mongolian (Khalkha dialect)}%, French}
\end{resdescription}
\end{ressec}

\begin{ressec}{Computer skills}
	\begin{resdescription}
		\resdescitem{Typesetting:}{{\fontspec{Palatino Linotype}\TeX} ({\fontspec{Palatino Linotype}\XeLaTeX} and {\fontspec{Palatino Linotype}\LaTeX})}
		\resdescitem{Markup and style languages:}{HTML, CSS}
		\resdescitem{Programming languages:}{C/C++, PHP, Python, Java}
		\resdescitem{Text editing:}{Vim}
		\resdescitem{Operating systems:}{(Arch, Gentoo) Linux, FreeBSD, Mac~OS~X}
		\resdescitem{Web publishing:}{Jekyll, MediaWiki, Wordpress}
		\resdescitem{Database systems:}{MySQL, Oracle~DBMS}
	\end{resdescription}
\end{ressec}

\vfill

\hfill {\footnotesize 2013·11·12}

%\end{multicols}
\end{document}
