%! TEX program = lualatex

\documentclass[11pt,a4paper]{moderncv}        % possible options include font size ('10pt', '11pt' and '12pt'), paper size ('a4paper', 'letterpaper', 'a5paper', 'legalpaper', 'executivepaper' and 'landscape') and font family ('sans' and 'roman')

% \tolerance=99999
\usepackage{microtype}

\usepackage[unicode]{hyperref}

% moderncv themes
\moderncvstyle{banking}                             % style options are 'casual' (default), 'classic', 'oldstyle' and 'banking'
\renewcommand*{\makeheaddetailssymbol}{\qquad}

\moderncvcolor{blue}                               % color options 'blue' (default), 'orange', 'green', 'red', 'purple', 'grey' and 'black'
% \definecolor{turonBlue}{HTML}{004C7F}
% \definecolor{darkscarlet}{rgb}{0.34, 0.01, 0.1}
% \definecolor{Burgundy}{RGB}{144, 0, 32}
% \definecolor{color0}{rgb}{0,0,0}% black
\definecolor{color1}{HTML}{004C7F}% light blue
% \definecolor{color2}{rgb}{1.0, 0.01, 0.1}% dark grey
\renewcommand*{\namefont}{\Huge\mdseries\upshape}
%\renewcommand{\familydefault}{\sfdefault}         % to set the default font; use '\sfdefault' for the default sans serif font, '\rmdefault' for the default roman one, or any tex font name
%\nopagenumbers{}                                  % uncomment to suppress automatic page numbering for CVs longer than one page
\patchcmd{\makehead}
  {\setlength{\makeheaddetailswidth}{0.8\textwidth}}
  {\setlength{\makeheaddetailswidth}{0.7\textwidth}}
  {}
  {}
\makeatletter
\patchcmd{\makehead}{%search
  \flushmakeheaddetails\@firstmakeheaddetailselementtrue\\\null}{%replace
  \flushmakeheaddetails\@firstmakeheaddetailselementtrue\par\vspace{-0.9\baselineskip}\null}{%success
  }{%failure
  }
\makeatother

%\usepackage[scale=0.75]{geometry}
\usepackage{geometry}
%\setlength{\hintscolumnwidth}{3cm}                % if you want to change the width of the column with the dates
%\setlength{\makecvtitlenamewidth}{10cm}           % for the 'classic' style, if you want to force the width allocated to your name and avoid line breaks. be careful though, the length is normally calculated to avoid any overlap with your personal info; use this at your own typographical risks...

% Wait for Biblatex 2; http://tex.stackexchange.com/questions/55256
%\DeclareSortingScheme{noneyear}{
% \sort{\citeorder}
% \sort{\field{year}}
%}

\usepackage{marginnote,needspace}

\makeatletter
\renewcommand\sectionfont{\color{color1}\large}
\renewcommand*{\sectionstyle}[1]{{%
	\needspace{1\baselineskip}%
	% \reversemarginpar\marginnote{\fontspec{Symbola}⋄}\normalmarginpar%
	\hspace{-0.5em}%
	{\sectionfont\MakeLowercase{\textsc{#1}}}}}
\renewcommand*{\section}[1]{%
  \par\addvspace{3.5ex}%
  \phantomsection{}% reset the anchor for hyperrefs
  \addcontentsline{toc}{section}{#1}%
  \sectionstyle{#1}%
  \par\nobreak\addvspace{1.0ex}\@afterheading}

\renewcommand\subsectionfont{\color{color1}}
\renewcommand*{\subsectionstyle}[1]{{%
	\parbox[t]{\textwidth}{\subsectionfont\textit{#1}}}}
\renewcommand*{\subsection}[1]{%
  \addvspace{1.5ex}%
  \phantomsection{}% reset the anchor for hyperrefs
  \addcontentsline{toc}{subsection}{#1}%
  \subsectionstyle{#1}%
  \par\nobreak\addvspace{2.0ex}\@afterheading}
\makeatother


\usepackage[backend=biber, style=authoryear, eventdate=comp, sorting=none]{biblatex}
\bibliography{publications}
\renewcommand{\mkbibnamefamily}[1]{\textsc{#1}}
\renewcommand{\subtitlepunct}{:\ }


\usepackage[silent]{fontspec}
\usepackage[silent,nil,bidi=basic]{babel}

\babelprovide[import=en-GB,main]{english}
\babelprovide[import=cy]{welsh}
\babelprovide[import=he]{hebrew}

% \WarningFilter{fontspec}{is using the default features for language}

\babelfont[hebrew]{rm}[Script=Hebrew, ItalicFont={Days-Nights}, ItalicFeatures={Scale=0.95}]{RacRaze}

\usepackage{fontspec}
%\usepackage{xltxtra}
% \usepackage{bidi}
% \babelfont{rm}[
		% ItalicFont = {Gentium Plus Italic},
		% BoldFont       = {Gentium Basic Bold},
		% BoldFeatures   = {%
			% SmallCapsFont=JuniusX,
			% SmallCapsFeatures={Letters=SmallCaps},
			% SmallCapsFeatures={FakeBold=2},
		% },
		% BoldItalicFont = {Gentium Basic Bold Italic},
		% BoldItalicFeatures   = {%
			% SmallCapsFont=JuniusX,
			% SmallCapsFeatures={Letters=SmallCaps},
			% SmallCapsFeatures={FakeBold=2},
		% },
	% ]{Gentium Plus}
\setmainfont{Gentium Plus}
%\setmainfont{Gentium Book Basic}
\setmonofont[Scale=0.9]{PragmataPro}

% \newcommand{\Hebrew}[1]{\bgroup\fontspec[Script=Hebrew, ItalicFont={Days-Nights}, ItalicFeatures={Scale=0.95}]{Rutz_OE} \RL{#1}\egroup}
% \newcommand{\BiblicalHebrew}[1]{\bgroup\fontspec[Script=Hebrew]{SBL Hebrew} \RL{#1}\egroup}
\newcommand{\BiblicalHebrew}[1]{\bgroup\fontspec[Script=Hebrew, Renderer=Harfbuzz]{SBL Hebrew}\textdir TRT #1\egroup}
\newcommand{\Hebrew}[1]{\BiblicalHebrew{#1}}
\newcommand{\texfont}{\fontspec{Palatino Linotype}}

\usepackage[super]{nth}

\newcommand{\foreign}[1]{\emph{#1}}

\name{Júda}{Ronén\vspace{0.5ex}}
%\title{}                               % optional, remove / comment the line if not wanted
\address{44/C/4 Mekor Ĥajim St.}{Jerusalem, 9346552}{Israeli Post}% optional, remove / comment the line if not wanted; the "postcode city" and "country" arguments can be omitted or provided empty
%\phone[mobile]{+972-54-6509361}                   % optional, remove / comment the line if not wanted; the optional "type" of the phone can be "mobile" (default), "fixed" or "fax"
\phone[mobile]{+972-52-5587769}                   % optional, remove / comment the line if not wanted; the optional "type" of the phone can be "mobile" (default), "fixed" or "fax"
\phone[fixed]{+972-2-6419913}
\email{foo@digitalwords.net}
\homepage{me.digitalwords.net}
%\social[github]{rwmpelstilzchen}
\social[stackexchange]{3978580}
%\extrainfo{additional information}                 % optional, remove / comment the line if not wanted
\photo[64pt][0.4pt]{qr.png}                       % optional, remove / comment the line if not wanted; '64pt' is the height the picture must be resized to, 0.4pt is the thickness of the frame around it (put it to 0pt for no frame) and 'picture' is the name of the picture file
%\quote{Some quote}                                 % optional, remove / comment the line if not wanted


\newcommand{\cvdate}{\the\year-\the\month-\the\day}
\fancyfoot[l]{\color{color2}[\cvdate]}

\begin{document}

\makecvtitle

\section{Education}

\subsection{Academic}

\cventry{2023}
{Hebrew University of Jerusalem}{Ph.D.\ in linguistics}{Jerusalem}{}{}%{2016–}
\cventry{2015}
{Hebrew University of Jerusalem}{M.A.\ in Linguistics, \foreign{magna cum laude}}{Jerusalem}{}{}%{2008–2015}
\cventry{2007}
{Hebrew University of Jerusalem}{B.A.\ in Linguistics and B.Sc.\ in Computer Science}{Jerusalem}{}{}%{2004–2007}



\subsection{Spoken Welsh courses}

\cventry{2019}
{Ysgol y Gymraeg, Prifysgol Caerdydd}{Cwrs dwys haf (intensive summer school)}{Caerdydd}{}{}
\cventry{2009}
{Coleg Meirion-Dwyfor}{Ysgol haf dysgu Cymraeg (Welsh language summer school)}{Pwllheli}{}{}

% \vspace{1\normalbaselineskip}



\subsection{Ph.D.\ thesis}

\cvitem{Title}{\emph{Grammar and textuality: A structural linguistic study of text-types Modern Welsh}}
\cvitem{Advisors}{%
	Prof.\ Erich \textsc{Poppe} (Philipps-Universität Marburg) and
	Dr.\ Lea \textsc{Sawicki} (Hebrew University of Jerusalem)
}
% \cvitem{Abstract}{A structural text-linguistic inquiry of Modern Welsh, describing the linguistic properties of sub-textual constituents (\emph{textemes}) and their interrelations. The main corpus consists of the textually diverse writings of Kate Roberts.}
\cvitem{Abstract}{A linguistic study of the means of text construction and organisation in Literary Modern Welsh, focussing on three textual components: anecdotes in autobiographical texts, reporting of speech in short stories, and stage directions in plays. The main corpus consists of the textually diverse writings of Kate Roberts.}
\cvitem{Language}{English}



% \vspace{1\normalbaselineskip}
\subsection{Master’s thesis}

\cvitem{Title}{\emph{%
	%{se ðe ís soð wysdom~· {\fontspec{Junicode}⁊} sawla líf}
	A textual-structural study of the demonstrative pronouns in Old English:
	Classification, characterisation and description
	of the examples in Ælfric’s \emph{Lives of Saints}}}
\cvitem{Advisor}{Dr.\ Nimrod \textsc{Barri} OBM (Hebrew University of Jerusalem)}
\cvitem{Abstract}{
Description of the syntactical structures in Old English, in which the demonstrative pronouns (\textit{se} and \textit{ðis}) occur independently of a following nominal phrase: structural classification, characterisation and analysis of their internal structure and their function in the text by a comprehensive examination of the examples in the corpus, Ælfric’s \emph{Lives of Saints}.
}
\cvitem{Language}{Hebrew}%, as required by the Hebrew University of Jerusalem for MA theses}
\cvitem{URL}{https://digitalwords.net/media/mathesis/mathesis.pdf}
%{\url{https://digitalwords.net/mathesis} (PDF and \LaTeX\ source)}


\section{Academic interests}
\cvlistdoubleitem
	{Welsh and Old English}
	{Micro- and macro-syntax}
\cvlistdoubleitem
	{Text linguistics}
	{Linguistic analysis of translation}
\cvlistdoubleitem
	{Linguistic analysis of narrative}
	{General linguistics}



\section{Conference presentations}

\subsection{Talks}

\cventry{2023}
	{The author as a weaver:}
	{\nth{17} International Congress of Celtic Studies}
	{Utrecht}
	{}
	{\emph{The grammatical make-up of the interface between narrative and dialogue in Welsh}}

\cventry{2019}
	{Dyna fy mywyd:}
	{\nth{16} International Congress of Celtic Studies}
	{Bangor}
	{}
	{\emph{Text-linguistic analysis of autobiographical anecdotes}}

\cventry{2018}
	{Wyddoch-Chi-Pwy:}
	{\nth{3} Poznań Conference of Celtic Studies}
	{Poznań}
	{}
	{\emph{Harri Potter and the Sociopragmatics of Second Person}}

\cventry{2018}
	{Wyddoch-Chi-Pwy:}
	{\nth{4} Usage-Based Linguistics Conference}
	{Tel Aviv}
	{}
	{\emph{Harri Potter and the Sociopragmatics of Second Person}}

\cventry{2015}
	{llygaid [sydd] ganddynt, ac ni welant:}
	{\nth{15} International Congress of Celtic Studies}
	{Glasgow}
	{}
	{\emph{Mediating senses through translation choices}}

\cventry{2012}
	{\fontspec{Gentium Plus}na{\fontspec{Gentium Plus}ʿ}ăśɛ wə-nišmå{\fontspec{Gentium Plus}ʿ}:}
	{Studientag in the Honour of Ariel Shisha-Halevy}
	{Jerusalem}
	{}
	{\emph{A linguistic study of a translator’s choices — the verb \emph{šm{\fontspec{Gentium Plus}ʿ}} in the Welsh translation of the Pentateuch} (presented in Hebrew; \Hebrew{\BiblicalHebrew{נַעֲשֶׂ֥ה וְנִשְׁמָֽע}: עיון בלשני בבחירותיו של מתרגם — הפועל „\BiblicalHebrew{שָׁמַע}” בתרגום התנ״ך הוולשי}); invited}
	% at a \emph{Studientag} in the honour of Ariel Shisha-Halevy, on the occasion of his retirement; the Hebrew University of Jerusalem, the Department of Linguistics.

\cventry{2011}
	{Ni a wnawn, ac a wrandawn:}
	{\nth{14} International Congress of Celtic Studies}
	{Maynooth}
	{}
	{\emph{William Morgan’s choices in his 1588 Welsh translation of the Pentateuch — the case of the Hebrew verb \emph{šm{\fontspec{Gentium Plus}ʿ}} (‘hear’)}}

\cventry{2009}
	{Between Narrative and Dialogue:}
	{Cynhadledd Ryngwladol Edward Lhuyd}
	{Aberystwyth}
	{}
	{\emph{Syntactical features of signalling speech within narrative in Modern Welsh}}

	

\subsection{Posters}

\cventry{2019}
	{A typological study of lateral fricatives}
	{Phonology and Phonetics Afternoon}
	{Jerusalem}
	{}
	{}

\cventry{2018}
	{The demonstrative pronouns in Old English:}
	%{The Sixth International Conference for Graduate Students on Diverse Approaches to Linguistics}
	{\parbox[t]{0.8\textwidth}{\nth{6} International Conference for Graduate\\students on Diverse Approaches to Linguistics (IGDAL)}}
	{Haifa}
	{}
	{\emph{A textual-structural analysis}}



\section{Teaching}

See under \emph{Employment} below.



\section{Conference organisation}

\cventry{2019}
	{Member of the organising committee}
	{\parbox[t]{0.8\textwidth}{\nth{7} International Conference for Graduate\\students on Diverse Approaches to Linguistics (IGDAL)}}
	{Jerusalem}
	{}
	{}



\subsection{Chairing}

\cventry{2023}
	{Middle and Modern Welsh morphology and pragmatics session}
	{\parbox[t]{0.8\textwidth}{\nth{17} International Congress of Celtic Studies}}
	{Utrecht}
	{}
	{}

\cventry{2019}
	{Noon session}
	{\parbox[t]{0.8\textwidth}{\nth{7} International Conference for Graduate\\students on Diverse Approaches to Linguistics (IGDAL)}}
	{Jerusalem}
	{}
	{}

\section{Research projects}

\cventry{2008–2010}
	{Šenoute’s Rhetorical Syntax}
	{Hebrew University of Jerusalem / Israel Science Foundation}
	{Jerusalem}
	{}
	{Assistant to PI Prof. Ariel \textsc{Shisha-Halevy}}



\section{Awards}

\cventry{2008}
	{Fraenkel award for outstanding papers in linguistics} % פרס ע״ש גד פרנקל
	{Hebrew University of Jerusalem}
	{Jerusalem}
	{}
	{Paper title: \emph{Between narrative and dialogue~— syntactical features of speech within narrative in Modern Welsh} (written in Hebrew); advisor: Prof. Ariel \textsc{Shisha-Halevy}}



%%% \section{Publications}
%%% 
%%% \nocite{*}
%%% \printbibliography[keyword=edit-text, heading=subbibliography, title={Editing, collecting, digitising and typesetting texts}]
%%% \printbibliography[keyword=edit-music, heading=subbibliography, title={Editing, collecting, digitising and engraving music}]
%%% \printbibliography[keyword=language-instruction, heading=subbibliography, title={Language instruction}]



\section{Employment}

\cventry{2019–2021}
	{Developing and teaching courses on the Welsh language}
	{Hebrew University of Jerusalem}
	{Jerusalem}
	{}
	{\emph{Introduction to the structure of Modern Welsh} (\#41581) and \emph{Topics in Modern Welsh grammar} (\#41582)}

\cventry{2014}
	{Developing and teaching a course on Esperanto}
	{Ofek School for the Gifted}
	{Jerusalem}
	{}
	{}

\cventry{2007–2010}
	{Webmaster and web developer}
	{Ofek School for the Gifted}
	{Jerusalem}
	{}
	{Developing, designing and maintaining a website, using MediaWiki, and school web applications (course selection and feedback), using PHP and MySQL}

\cventry{2007–2009}
	{Typesetter}
	{Journal d’Analyse Mathématique and Israel Journal of Mathematics}
	{Jerusalem}
	{}
	{Typesetting and correcting proofread articles in mathematics, using {\texfont\LaTeX}}



\section{Language skills}

\cvitem{Oral and written}
	{Hebrew (native), English, Welsh, Norwegian, Esperanto}

% \cvitem{Fluent}
	% {Hebrew (native), English}
% \cvitem{Main research languages}
	% {Modern Welsh, Old English}
% \cvitem{Talking and reading}
	% {Norwegian, Esperanto}

\cvitem{Written}
	{Old English, Middle Welsh, Old Norse, Danish, Swedish}

\cvitem{Basic knowledge}
	{%
		Egyptian (Coptic and Ancient),
		Finnish*,
		German*,
		Gothic,
		Greek,
		Irish,
		Japanese,
		Latin,
		Malayalam,
		Mongolian (Khalkha dialect),
		Palestinian Arabic*,
		Tamil
		{\normalfont\small (took academic courses, except the ones marked with an asterisk)}
	}



\section{Technology skills}

% \renewcommand{\doubleitemcolumnwidth}{0.5\textwidth-1em}
% \cvdoubleitem

\cvline{Bibliography management}{Bib{\texfont\LaTeX}}
\cvline{Database systems}{SQLite, MariaDB/MySQL} % TODO: change structure to something more general %, Oracle DBMS
\cvline{Data interchange}{JSON, YAML}
\cvline{Data visualisation}{\mbox{\textsc{pgf}/T\textit{ik}Z}, Inkscape}
\cvline{Linguistics}{Praat, ELAN}
\cvline{Markup and style languages}{HTML, XML (+TEI), CSS}
\cvline{Operating systems}{Linux, FreeBSD, Mac~OS~X}
\cvline{Programming languages}{Python, Lua, C/C++, PHP, Java}
\cvline{Text editing}{NeoVim}
\cvline{Typesetting}{{\texfont\LaTeX}, LilyPond}
\cvline{Version control}{Git}
\cvline{Web publishing}{Pelican, Jekyll, MediaWiki, Wordpress}



%%% \section{Music skills}
%%% 
%%% \cvitem{Plucked string instruments}{Mandolin, guitar (Renaissance lute music), ukulele}
%%% \cvitem{Fipple flutes}{Irish whistle, ocarina, recorder}
	%%% TODO: flute
%%% \cvitem{Keyboard}{Piano, melodica}
%%% \cvitem{Percussion}{Javanese gamelan instruments (member of the \href{https://www.facebook.com/jerusalem.gamelan/}{Jerusalem Gamelan})}

\end{document}
