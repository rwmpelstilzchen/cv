\documentclass[a4paper]{article}
%\usepackage[top=3.0cm,bottom=4.3cm,left=3.0cm,right=3.0cm]{geometry}
\usepackage[left=6.5cm,top=2.5cm,right=2.5cm,bottom=2.5cm,nohead,nofoot]{geometry}

\usepackage{longtable, tabu}

\usepackage{fontspec}
\usepackage{xltxtra}
\usepackage{bidi}

\newcommand{\insertbox}[1]{%
  \llap{\smash{\parbox[t]{3cm}{#1}}}%
}

\usepackage{adjustbox}
\renewcommand\insertbox[1]{%
    \adjustbox{minipage=[t]{#1},cfbox=red,set height=0pt,set depth=0cm,lap=-\width-1cm}%
}

% Section of resume: one argument: Name of section
%\newenvironment{ressec}[1]{\begin{tabular}{p{1.4in}p{4.2in}} {\bfseries\fontspec{Fontin} \textsc{#1}} & }{\end{tabular}}
%\newenvironment{ressec}[1]{\begin{tabular}{p{1.4in}p{4.2in}} {\fontspec[Color=203040FF]{Fontin SmallCaps} \textsc{#1}} & }{\end{tabular}}
\newenvironment{ressec}[1]{\insertbox{3.5cm}{\fontspec[Color=203040FF]{Fontin SmallCaps} \textsc{#1}}}{}

% Itemize environment for resume. Better layout
\newenvironment{resitemize}{\setlength\parindent{-20pt}\begin{tabular}[t]{p{-0.0in}p{5.3in}}}{\end{tabular}\setlength\parindent{0pt}}
% Itemize item for resume
\newcommand{\resitem}{$\sbullet$ & }

\usepackage{ruby}
\renewcommand{\rubysize}{0.64}
\renewcommand{\rubysep}{0.10ex} 


\newenvironment{resdescription}
%	{\begin{tabular}[t]{p{2in}p{5.3in}}}
%	{\begin{tabular}[t]{p{1.1in}p{3in}}}
%	{\end{tabular}}
	%{\setlength\parindent{-20pt}}
	{\setlength\parindent{-20pt}}
	{\setlength\parindent{0pt}}
\newcommand{\resdescitem}[2]{{#1}\hspace{10pt}#2

}

% Job item under, say, an 'experience' section. 2 args: job (to be aligned left) and date
% in parenthesis and aligned right. All of it is bold.
\newcommand{\resjob}[2]{\parbox{5.65in}{\vspace{0.05in}\textbf{\textsf{#1}} \hfill \textbf{\sf(#2)}\vspace{0.05in}}}

% description of job, under a job item. the argument is the text you want
\newcommand{\resjobinfo}[1]{\parbox{5.65in}{#1}}

% Small bullet, for use in text mode. Nicer than full sized bullet, in my opinion
\newcommand{\sbullet}{\,\begin{picture}(1,1)(0,-3)\circle*{3}\end{picture}\ }




\newcommand{\Hebrew}[1]{\RL{\fontspec[Language=Hebrew, Script=Hebrew]{SBL Hebrew}#1}}

%\hyphenpenalty=5000
\tolerance=99999

\usepackage{marginnote, graphicx}

\setlength\parskip{\bigskipamount}
\setlength\parindent{0pt}
%\renewcommand{\footnoterule}{\vskip0.3cm}

\IfFileExists{url.sty}{\usepackage{url}}
{\newcommand{\url}{\texttt}}

%\setromanfont{Palatino}
\setromanfont{Optima LT Std}
\setmonofont{LMTypewriter10 Regular}

%\newfontfamily\HebrewFont[Script=Arabic,HyphenChar={-}]{Bitstream Cyberbit}

%\renewcommand{\L}[1]{{\LR{\rmfamily\nonfrenchspacing{}#1}}}
%\newcommand{\R}[1]{{\RL{\HebrewFont\frenchspacing{}#1}}}

\newcommand{\ndash}{\LAT{–}}
\newcommand{\mdash}{\LAT{—}}
%\newcommand{\br}{\linebreak[4]}

\renewcommand{\baselinestretch}{1.1}
\long\def\symbolfootnote[#1]#2{\begingroup%
\def\thefootnote{\fnsymbol{footnote}}\footnote[#1]{#2}\endgroup}

\newcommand{\markout}[1]{\underline{#1}}

\renewcommand{\emph}[1]{\textbf{#1}}
%\renewcommand{\emph}[1]{{\addfontfeatures{FakeBold=2}#1}}
